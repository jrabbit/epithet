\documentclass[a4paper,10pt]{article}


%opening
\title{The Netsukuku Civic Net}
\author{by\\\begin{small}The Netsukuku team                              \end{small}\\\textit{\begin{small}http://netsukuku.freaknet.org                                        \end{small}}}

\begin{document}

\maketitle
\pagebreak
Cities and Institutions are showing a growing interest in e-government
projects that are targetted in providing on line services to the citizens
through Internet.
For example, the following are candidates for such projects:
\begin{enumerate}
	\item 
School related activities (enrolments, cultural and educational programs).
\item News, local radio and televisions.	                         
\item  Guides for expositions and events, cultural and tourist information.	    
\item  Discussion forums, social and cultural activities.	                        
\end{enumerate}

Unfortunately, the net is not so readily available for everyone: 
DSL facilities aren't available in many cities and ISPs tend
to impose expensive rates/costs to grant it.

The creation of civic nets, where users can have free access to web sites
and to online services, surfing web sites devoted to institutional, educational,
tourist, commercial and social activities freely and without any cost is now
possible by Netsukuku.

Netsukuku's first goal is the implementation of a global mesh network, completely
independent from the Internet (even if compatible with), where single PC's communicating each other
through radio frequencies will become automatically integral part of
the net, keeping it alive and, at the same time, expanding it.

Netsukuku's special routing system allows:
\begin{enumerate}
	\item 
 Every user to go not only through the Net of their city but also through
  those located in all the other cities which, once they are interconnected,
  create a special kind of global Network, parallel and joined to Internet, that will
  indefinitely keep in expansion, completely without the support of any ISP.
\item 256 (max) hostnames to be recorded and supported by each node.
\end{enumerate}

Last but not least users will have the chance to get
through the Internet by sharing their Internet connectivity within the Netsukuku
network. The other users will gain Internet access, but they will be even able
to exploit the available bandwidth, like a single, large Internet connection.

\pagebreak
\begin{center}\verb|^_^|\end{center}

\end{document}
