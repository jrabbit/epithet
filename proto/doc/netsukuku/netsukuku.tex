%        File: netsukuku.tex
%     Created: Sun Nov 19 03:00 PM 2006 C
% Last Change: Sun Nov 19 03:00 PM 2006 C
%
% This file is part of Netsukuku
% (c) Copyright 2007 Andrea Lo Pumo aka AlpT <alpt@freaknet.org>
%
% This source code is free software; you can redistribute it and/or
% modify it under the terms of the GNU General Public License as published 
% by the Free Software Foundation; either version 2 of the License,
% or (at your option) any later version.
%
% This source code is distributed in the hope that it will be useful,
% but WITHOUT ANY WARRANTY; without even the implied warranty of
% MERCHANTABILITY or FITNESS FOR A PARTICULAR PURPOSE.
% Please refer to the GNU Public License for more details.
%
% You should have received a copy of the GNU Public License along with
% this source code; if not, write to:
% Free Software Foundation, Inc., 675 Mass Ave, Cambridge, MA 02139, USA.
%

\documentclass[a4paper]{article}
\usepackage{color,graphicx}
\usepackage{amsmath}
\usepackage{amsthm}
\usepackage{amssymb}
\usepackage{amsfonts}
\RequirePackage{ifpdf} % running on pdfTeX?
\ifpdf
\usepackage[pdftex,bookmarks=true,
		   bookmarksnumbered=false,
		   bookmarksopen=false,
		   colorlinks=true,
		   linkcolor=webred] {hyperref}
\definecolor{webgreen}{rgb}{0, 0.5, 0} % less intense green
\definecolor{webblue}{rgb}{0, 0, 0.5} % less intense blue
\definecolor{webred}{rgb}{0.5, 0, 0}   % less intense red
\else
\newcommand{\href}[2]{ #1 }
\fi
\title{Netsukuku\\
{\small Close the world, \reflectbox{Open the next}}}
\author{http://netsukuku.freaknet.org}
	%AlpT (@freaknet.org)
\begin{document}
\maketitle

\begin{abstract}
Netsukuku is a P2P network system designed to handle massive numbers of nodes 
with minimal consumption of CPU and memory resources.  It can be used to build a
world-wide distributed, fault-tolerant, anonymous, and censorship-immune network,
fully independent from the Internet.\\
Netsukuku does not rely upon any form of backbone router, internet service 
provider network, or any centralized system, although it may take advantage of
existing systems of this nature to augment unity and connectivity of the 
existing Netsukuku network. 

In this document, we will give a plain-english description of the theory 
behind the Netsukuku system, with a focus upon core concepts and capabilities.
\end{abstract}


\section{The old wired}

The Internet is a network maintained, operated, and controlled 
by large corporations and governments.  Each and every packet of data must 
traverse countless backbone routers via unending lengths of fiber, all 
under corporate or government ownership, and subject to their exclusive control.

The Internet Service Providers provide connectivity to the lowest rank of
this pyramid, the end user.
This is far removed from the ideal of a global user-based and decentralized 
network, depending upon the ISPs, both putting them in a position of power 
over the data received by the end users, and making them a single point of
failure for those who they provide service to.
The people can join the net only in accordance with the restrictive conditions
and terms imposed by the ISPs, and subject to the filters and restrictions
placed upon them, often even without their knowledge.

Today, the Internet represents the ultimate means of access to information, 
knowledge and communication.  More than 1 billion people\cite{mwmgstats}
can connect to this massive and immensely valuable, yet fundamentally 
proprietary and controlled, network. As impressive a statistic as this may
be, the remaining 5 billion, lacking the economic resources necessary to 
assemble the necessary infrastructure or subscribe to what services are 
already available, are still waiting for the multinationals to supply a 
service within their reach.  In this modern day, their lack of connectivity 
is more than an inconvenience; without this tool available to them, they 
are at a massive economic and educational disadvantage to those of us 
who possess it. \\

The Internet was designed to be a secure, distributed, and failure-resistant
communication system of such quality that it would be appropriate for military
usage. But now, paradoxically, as even the rigidly controlled military system
proposed held distribution and redundancy in high regard, ISPs has the power
to completely cut off even entire nations by simply choosing to provide
their services only on favourable terms\cite{digitaldivide}, granting
them an obscene amount of power, which is already commonly taken
advantage of in no small way for profit. This prolific centralization
also enables more sinister abuses of power, such as censorship of all
dissident thought.
This becomes even more worrying in light of the fact that any corporation who
owns a single backbone router may view and modify a massive amount of the
total data flowing through the Internet, even when it begins and ends at 
points outside of their control. Thanks the non anonymous nature of the 
Internet, the ISP can easily trace all traffic going through their servers
back to its point of origin\dots Additionally, authorities with access 
to ISP data may use and have used the system as a means to identify and 
persecute locals who voice dissent\cite{ipdindymedia}.
This state of affairs is widely known, but few realise how close to home
this practice truly is. Internet traffic is already routinely sniffed by
countless national governments, and there even exists a small industry
producing hardware and software to process the massive streams of data
captured.\\

Throughout the history of the Internet, its centralised 
structure has been the model for many subsidiary systems, such as the
Domain Name System. DNS servers are managed by many of the same organizations
responsible for providing Internet service (beyond simple consumer ISPs), with 
registration of a domain upon these servers a privilege provided by InterNIC, 
a United States controlled company, who grants registrars the right to sell 
single domain names to end users.  This system is subject to all of the same 
issues as the Internet itself, if not more, resulting from its even greater
centralization, and the reliance upon it for locating servers by almost all
users of the Internet.\\

It is utopistic to bring more accessibility, freedom of expression and
privacy to the Internet as long as these efforts rely on the existing
system. But a remaining option is still allowed: to conceive a system to
rebuild from the ground up with a network, requiring support from none
but those that use it. Such a system is inherently censorship-immune:
as without any form of centralized backbone,  any given node is unable
to have any sort of widespread effect upon data originating elsewhere,
or even form a clear idea of the content of data sent to it. Netsukuku
is designed to build such a system.


\section{The Netsukuku wired}

The Netsukuku network is composed of nearby computers directly linked each other, 
and thus has no dependence upon the Internet, or indeed any existing network.
The system augments level 3 of the OSI model with its own true distributed routing protocol.
Netsukuku's distributed nature is emulated by the core services that are built upon 
it to replace those with similar centralization problems to the Internet, such as
the previously discussed DNS, which is replaced by the introspectively-named 
ANDNA (A Netsukuku Domain Name Architecture)\cite{andnadoc}.

\subsection{Gandhi}

The most notable characteristic of Netsukuku is its fully distributed self-management.
The network dynamically configures itself without any external interventions, or any 
form of central organizing authority, something commonly believed to be infeasible,
if not outright impossible.  
All nodes share the same privileges, each making a contribution to sustain and expand
Netsukuku.
Of particular interest is Netsukuku's increase in efficiency proportional to the 
number of well-connected nodes in the network, meaning that more users will typically
lead to even lower latencies and greater bandwidth available to all.
This is the exact opposite of the current system, where more users simply add stress 
to the system, leading to long response times and slow data transfers.  
This makes the network almost self-improving, as each user of the network has incentive
to improve the network's quality.  Even he who consumes massive amounts of bandwidth
stands only to benefit from adding more interconnections of greater quality to more
nearby nodes, substantially improving the network for all.

This total decentralisation and distribution allows Netsukuku to be neither controlled
nor destroyed: the only way to manipulate or demolish it is to knock physically down 
each single node composing the network, making any form of attack or takeover attempt
completely infeasible.

\subsection{No name, no identity}

Netsukuku allows anyone, in any place, at any moment, to connect directly to the network
without need for paperwork or subscriptions. 
All the elements of the net are highly dynamic; nodes can come and go at will, needing to
retain no inherent identifying characteristics, making identity and even route immensely
malleable concepts.  The IP address identifying a computer is chosen randomly, making it
impossible to associate it with a particular location.  Furthermore, because the routes 
are composed by a huge number of nodes, it becomes a wholly infeasible task to trace
a specific node by its traffic.  Finally, traffic is protected by a strong cryptographic
layer\cite{carciofo}, which guarantees unparalleled security and anonymity for any connection.

\subsection{So, what is it?}

Netsukuku is a peer to peer or mesh network built on top of it's own dynamic routing protocol.
While currently there are many dynamic routing protocols, most are, unlike Netsukuku, incapable
of managing networks of significant size.  To continue comparison, Internet backbone routers
are managed by another set of functionally similar protocols, including OSPF, RIP, and BGP.
They use classical graph algorithms designed to find out the best path to reach a node in a 
given net-graph, making for reasonably efficient routing.  Unfortunately, all of these 
protocols must consume massive amounts of computational resources to function on a network
of the scale of the Internet, and must exist on special dedicated machines. 
So great are the physical requirements of these unique (usually even purpose-built) machines,
that decentralization is not only politically infeasible, but also economically impossible.

% Ralith: This paragraph needs a more in-depth and meaningful explanation.
%% Alpt: it's better to forward the reader to the appropriate documents
The Netsukuku protocol structures the topology of the network in different
layers of a compact hierarchy\cite{ntktopology}. The  
QSPN\cite{qspndoc} algorithm, designed for this
specific situation, is then used to determine routes. Since the topology is
characterized by an high degree of self-similarity, only the basic pattern must be stored.
This compression level grants the ability to store the entire network map in
just few kilobytes. On the other hand, the QSPN algorithm must be executed not by 
any central routers, but instead by the nodes of the network.  The component nodes perform this 
duty simply by generating, propagating, and parsing Tracer Packets (TPs), an activity 
that consumes very few computational resources.\\
For more information, please refer to the technical
documentation: \cite{ntktopology}, \cite{qspndoc}.

\subsection{The wireless}

The cheapest and most convenient medium to establish physical connections between 
typical urban nodes is radio, incarnate as WiFi and similar technologies.  
In a scenario of widespread adoption, a new Netsukuku user need do little more
than install a transceiver within range of other local nodes, linking themselves
into the network, and configure their computer to take advantage of it. 
Today there exist a wide variety of WiFi technologies and similar which allows 
wireless connectivity between nodes even several kilometers distant.  
Even with common consumer technologies, an entire city can be easily covered
by placing a single node in each neighbourhood.

Unfortunately, there will inevitably remain cases where geography or
distance prevent a direct radio link. In these situations, or even 
in many cases where a long range radio link is feasible, a high-bandwidth 
low-latency connection more along the lines of a fiber bundle can be
highly desirable, perhaps for connecting distant cities.  
However, such a solution is extremely costly, and is out of reach 
of a typical grassroots effort.  If and when Netsukuku becomes 
widely prolific, such projects might be sponsored by cities or
governments, but in the meantime, we are unlikely to see much of that.  
Therefore, as a stopgap measure, it is possible to replace missing 
physical links by tunneling over the Internet\cite{inetdoc}, a practice 
that should be discouraged in the long term, but which makes a global
Netsukuku network vastly more feasible in the immediate future.

% TODO: Conclusion

%%%%%%%%%%%%%%%%
% Bibliography %
%%%%%%%%%%%%%%%%

\begin{thebibliography}{99}
	\bibitem{ntksite} Netsukuku website:
		\href{http://netsukuku.freaknet.org/}{http://netsukuku.freaknet.org/}
	\bibitem{qspndoc} QSPN document:
		\href{http://netsukuku.freaknet.org/doc/main\_doc/qspn.pdf}{http://netsukuku.freaknet.org/doc/main\_doc/qspn.pdf}
	\bibitem{ntktopology} Netsukuku topology document:
		\href{http://netsukuku.freaknet.org/doc/main\_doc/topology.pdf}{http://netsukuku.freaknet.org/doc/main\_doc/topology.pdf}
	\bibitem{andnadoc} ANDNA document:
		\href{http://netsukuku.freaknet.org/doc/main\_doc/andna.pdf}{http://netsukuku.freaknet.org/doc/main\_doc/andna.pdf}
	\bibitem{carciofo} Carciofo NTK\_RFC:
		\href{http://lab.dyne.org/Ntk\_carciofo}{http://lab.dyne.org/Ntk\_carciofo}
	\bibitem{inetdoc}  Internet and Netsukuku:
		\href{http://netsukuku.freaknet.org/doc/main\_doc/inetntk.pdf}{http://netsukuku.freaknet.org/doc/main\_doc/inetntk.pdf}

	\bibitem{ccfilters} CNET article on the legality of Comcast's filtering:
		\href{http://www.cnet.com/8301-13739\_1-9769645-46.html}{http://www.cnet.com/8301-13739\_1-9769645-46.html}

	\bibitem{mwmgstats} Miniwatts Marketing Group Internet Statistics:
		\href{http://www.internetworldstats.com/stats.htm}{http://www.internetworldstats.com/stats.htm}


	\bibitem{ipdindymedia}  Second Indymedia Server Seized in UK Within a Year:
		\href{http://yro.slashdot.org/article.pl?sid=05/06/28/0113237\&tid=153\&tid=158\&tid=149\&tid=17}{http://yro.slashdot.org/article.pl?sid=05/06/28/0113237\&tid=153\&tid=158\&tid=149\&tid=17} \\
                 CBS on China's Internet Censorship:
		\href{http://www.cbsnews.com/stories/2002/12/03/tech/main531567.shtml}{http://www.cbsnews.com/stories/2002/12/03/tech/main531567.shtml}

        \bibitem{digitaldivide} The "Digital Divide" refers to the gap
        between the developed world and the developing world about the
        effective access to digital and information technology:
 		\href{http://www.digitaldivide.org/dd/index.html}{http://www.digitaldivide.org/dd/index.html}
\end{thebibliography}
\newpage

\begin{center}
\verb|^_^|
\end{center}
\end{document}
